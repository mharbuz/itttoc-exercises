\begin{enumerate}

    \item [1.22]
          In certain programming languages, comments appear between delimiters such as \slash\# and \#\slash .Let $C$ be the language of all valid delimited comment strings. A member of $C$ must begin with \slash \# and end with \# \slash but have no intervening \# \slash. For simplicity, assume that the alphabet for $C$ is $\Sigma=\{a,b,\text{/},\text{\#}\}$.
          \begin{enumerate}
              \item Give a DFA that recognizes $C$.
                    \begin{figure}[H]
                        \centering
                        \begin{tikzpicture}[
                            ]
                            \node[state, initial] (q0) {};
                            \node[state, right of=q0] (q1) {};
                            \node[state, right of=q1] (q2) {};
                            \node[state, right of=q2] (q3) {};
                            \node[state, accepting, right of=q3] (q4) {};
                            \node[state, below of=q2] (q5) {};
                            \draw
                            (q0) edge[above] node{$\slash$} (q1)
                            (q1) edge[above] node{$\#$} (q2)
                            (q2) edge[loop above, above] node{$a,b,\slash$} (q2)
                            (q2) edge[above] node{$\#$} (q3)
                            (q3) edge[above] node{$\slash$} (q4)
                            (q0) edge[bend right, below left] node{$a,b,\#$} (q5)
                            (q1) edge[bend left, below left] node{$a,b,\slash$} (q5)
                            (q3) edge[bend left, below left] node{$a,b$} (q2)
                            (q3) edge[loop above, above] node{$\#$} (q3)
                            (q5) edge[loop below, below] node{$a,b,\slash,\#$} (q5);
                        \end{tikzpicture}
                        \caption{DFA recognizing $C$}
                    \end{figure}
              \item Give a regular expression that generates $C$.
                    $$\slash \#(a \cup b \cup \slash \cup (\#^\ast(a \cup b)))^\ast \# \slash $$
          \end{enumerate}
    \item [1.23]
          Let B be any language over the alphabet $\Sigma$. \\
          Prove that $B = B^+$ iff $BB \subseteq B$.

          Step 1\\
          Assume $B = B^+$. \\
          Let $x \in BB$. \\
          Then $x = uv$ for some $u,v \in B$. \\
          Since $B = B^+$, $u \in B^+$ and $v \in B^+$. \\
          Therefore $uv \in B^+$. \\
          Since $B^+ = B$, $uv \in B$. \\
          Therefore $BB \subseteq B$. \\
          \\
          Step 2\\
          Assume $BB \subseteq B$. \\
          Let $x \in B^+$. \\
          Then $x = uv$ for some $u,v \in B$. \\
          Since $BB \subseteq B$, $uv \in B$. \\
          Therefore $B^+ \subseteq B$. \\
          Since $B \subseteq B^+$ and $B^+ \subseteq B$ then $B = B^+$. \\
          Therefore $B = B^+$ iff $BB \subseteq B$. \\
    \item [1.24]
          A finite state transducer (FST) is a type of deterministic finite automaton whose output is a string and not just accept or reject. The following are state diagrams of finite state transducers $T_1$ and $T_2$.
          \begin{enumerate}
              \item $T_1$ on input $011$
                    \begin{table}[H]
                        \centering
                        \begin{tabular}{|c|c|}
                            \hline
                            States        & Output \\
                            \hline
                            $q1,q1,q1,q1$ & $000$  \\
                            \hline
                        \end{tabular}
                    \end{table}
              \item $T_1$ oninput $211$
                    \begin{table}[H]
                        \centering
                        \begin{tabular}{|c|c|}
                            \hline
                            States        & Output \\
                            \hline
                            $q1,q2,q2,q2$ & $111$  \\
                            \hline
                        \end{tabular}
                    \end{table}
              \item $T_1$ on input $121$
                    \begin{table}[H]
                        \centering
                        \begin{tabular}{|c|c|}
                            \hline
                            States        & Output \\
                            \hline
                            $q1,q1,q2,q2$ & $011$  \\
                            \hline
                        \end{tabular}
                    \end{table}
              \item $T_1$ oninput $0202$
                    \begin{table}[H]
                        \centering
                        \begin{tabular}{|c|c|}
                            \hline
                            States           & Output \\
                            \hline
                            $q1,q1,q2,q1,q2$ & $0101$ \\
                            \hline
                        \end{tabular}
                    \end{table}
              \item $T_2$ oninput $b$
                    \begin{table}[H]
                        \centering
                        \begin{tabular}{|c|c|}
                            \hline
                            States  & Output \\
                            \hline
                            $q1,q3$ & $1$    \\
                            \hline
                        \end{tabular}
                    \end{table}
              \item $T_2$ on input $bbab$
                    \begin{table}[H]
                        \centering
                        \begin{tabular}{|c|c|}
                            \hline
                            States           & Output \\
                            \hline
                            $q1,q3,q2,q3,q2$ & $1111$ \\
                            \hline
                        \end{tabular}
                    \end{table}
              \item $T_2$ oninput $bbbbbb$
                    \begin{table}[H]
                        \centering
                        \begin{tabular}{|c|c|}
                            \hline
                            States                 & Output   \\
                            \hline
                            $q1,q3,q2,q1,q3,q2,q1$ & $110110$ \\
                            \hline
                        \end{tabular}
                    \end{table}
              \item $T_2$ oninput $\epsilon$
                    \begin{table}[H]
                        \centering
                        \begin{tabular}{|c|c|}
                            \hline
                            States & Output     \\
                            \hline
                            $q1$   & $\epsilon$ \\
                            \hline
                        \end{tabular}
                    \end{table}
          \end{enumerate}
\end{enumerate}
