\begin{enumerate}

    \item [1.13]
          
          Let $F$ be the language of all strings over $\{0,1\}$ that do not contain a pair of $1$s that are separated by an odd number of symbols. Give the state diagram of a DFA with five states that recognizes $F$. (You may find it helpful first to find a 4-state NFA for the complement of $F$.) blab 
          
          $\overline{F} = \{w|w~ \text{contains a pair of }1\text{'s that are separated by an odd number of symbols}\}$
          
          \begin{figure}[H]
              \centering
              \begin{tikzpicture}
                  \node[state, initial] (q0) {};
                  \node[state, right of=q0] (q1) {};
                  \node[state, above of=q1] (q2) {};
                  \node[state, accepting, right of=q1] (q3) {};
                  \draw
                  (q0) edge[above] node{$1$} (q1)
                  (q0) edge[bend right, below] node{$1$} (q2)
                  (q2) edge[above] node{$1$} (q3)
                  (q2) edge[bend right, above] node{$1$} (q3)
                  (q3) edge[loop above] node{$0$} (q3);
              \end{tikzpicture}
              \caption{NFA with four states that recognizes $\overline{F}$}
          \end{figure}
    \item [1.14]
          Give regular expressions generating the languages of Exercise 1.6.
          \begin{enumerate}
              \item $\{w|w~ \text{begins with a }1\text{ and ends with a }0\}$
                    \begin{align*}
                        1\Sigma^*0
                    \end{align*}
              \item $\{w|w~ \text{contains at least three }1\text{'s}\}$
                    \begin{align*}
                        \Sigma^*1\Sigma^*1\Sigma^*1\Sigma^*
                    \end{align*}
              \item $\{w|w~ \text{contains the substring }0101\}$
                    \begin{align*}
                        \Sigma^*0101\Sigma^*
                    \end{align*}
              \item $\{w|w~ \text{has length at least }3\text{ and its third symbol is a }0\}$
                    \begin{align*}
                        \Sigma\Sigma0\Sigma^*
                    \end{align*}
              \item $\{w|w~ \text{starts with }0\text{ and has odd length, or starts with }1\text{ and has even length}\}$
                    \begin{align*}
                        0(\Sigma\Sigma)^* \cup 1\Sigma(\Sigma\Sigma)^*
                    \end{align*}
              \item $\{w|w~ \text{doesn't contain the substring }110\}$
                    \begin{align*}
                        (0 \cup 10)^\ast (1 \cup 111^\ast \cup \epsilon) = 
                        (0 \cup 10)^\ast 1^\ast
                    \end{align*}
                    in details:
                    
                    \begin{figure}[H]
                        \centering
                        \begin{tikzpicture}
                            \node[state, accepting, initial] (q1) {};
                            \node[state, accepting, right of=q1] (q2) {};
                            \node[state, accepting, right of=q2] (q3) {};
                            \node[state, right of=q3] (q4) {};
                            \draw
                            (q1) edge[above] node{$1$} (q2)
                            (q1) edge[loop above] node{$0$} (q1)
                            (q2) edge[above] node{$1$} (q3)
                            (q2) edge[bend right, above] node{$0$} (q1)
                            (q3) edge[above] node{$0$} (q4)
                            (q3) edge[loop above] node{$1$} (q3)
                            (q4) edge[loop above] node{$0,1$} (q4);
                        \end{tikzpicture}
                        \caption{DFA that recognizes the language}
                    \end{figure}
                    
                    \begin{figure}[H]
                        \centering
                        \begin{tikzpicture}
                            \node[state, initial] (q0) {start};
                            \node[state, accepting, right of=q0] (q1) {A};
                            \node[state, accepting, right of=q1] (q2) {B};
                            \node[state, accepting, right of=q2] (q3) {C};
                            \node[state, right of=q3] (q4) {D};
                            \node[state, below of=q2] (qend) {acc};
                            \draw
                            (q0) edge[above] node{$\epsilon$} (q1)
                            (q1) edge[above] node{$1$} (q2)
                            (q1) edge[loop above] node{$0$} (q1)
                            (q2) edge[above] node{$1$} (q3)
                            (q2) edge[bend right, above] node{$0$} (q1)
                            (q3) edge[above] node{$0$} (q4)
                            (q3) edge[loop above] node{$1$} (q3)
                            (q4) edge[loop above] node{$0,1$} (q4)
                            (q1) edge[above] node{$\epsilon$} (qend)
                            (q2) edge[left] node{$\epsilon$} (qend)
                            (q3) edge[left] node{$\epsilon$} (qend);
                        \end{tikzpicture}
                        \caption{GNFA that recognizes the language}
                    \end{figure}
                    
                    \begin{figure}[H]
                        \centering
                        \begin{tikzpicture}
                            \node[state, initial] (q0) {start};
                            \node[state, accepting, right of=q0] (q1) {A};
                            \node[state, accepting, right of=q1] (q2) {B};
                            \node[state, accepting, right of=q2] (q3) {C};
                            \node[state, below of=q2] (qend) {acc};
                            \draw
                            (q0) edge[above] node{$\epsilon$} (q1)
                            (q1) edge[above] node{$1$} (q2)
                            (q1) edge[loop above] node{$0$} (q1)
                            (q2) edge[above] node{$1$} (q3)
                            (q2) edge[bend right, above] node{$0$} (q1)
                            (q3) edge[loop above] node{$1$} (q3)
                            (q1) edge[above] node{$\epsilon$} (qend)
                            (q2) edge[left] node{$\epsilon$} (qend)
                            (q3) edge[left] node{$\epsilon$} (qend);
                        \end{tikzpicture}
                        \caption{GNFA - removed state D}
                    \end{figure}
                    
                    \begin{figure}[H]
                        \centering
                        \begin{tikzpicture}
                            \node[state, initial] (q0) {start};
                            \node[state, accepting, right of=q0] (q1) {A};
                            \node[state, accepting, right of=q1] (q2) {B};
                            \node[state, below of=q2] (qend) {acc};
                            \draw
                            (q0) edge[above] node{$\epsilon$} (q1)
                            (q1) edge[above] node{$1$} (q2)
                            (q1) edge[loop above] node{$0$} (q1)
                            
                            (q2) edge[bend right, above] node{$0$} (q1)
                            (q2) edge[bend left, right] node{$11^\ast\epsilon$} (qend)
                            (q1) edge[above] node{$\epsilon$} (qend)
                            (q2) edge[left] node{$\epsilon$} (qend);
                        \end{tikzpicture}
                        \caption{GNFA - removed state C}
                    \end{figure}
                    
                    \begin{figure}[H]
                        \centering
                        \begin{tikzpicture}
                            \node[state, initial] (q0) {start};
                            \node[state, accepting, right of=q0] (q1) {A};
                            \node[state, right of=q1] (qend) {acc};
                            \draw
                            (q0) edge[above] node{$\epsilon$} (q1)
                            (q1) edge[loop above] node{$0 \cup 10$} (q1)
                            
                            (q1) edge[bend left, above] node{$1 \cup 111^\ast\epsilon$} (qend)
                            (q1) edge[bend right, below] node{$\epsilon$} (qend);
                        \end{tikzpicture}
                        \caption{GNFA - removed state B}
                    \end{figure}
                    
                    \begin{figure}[H]
                        \centering
                        \begin{tikzpicture}
                            [node distance=6cm]
                            \node[state, initial] (q0) {start};
                            \node[state, right of=q0] (qend) {acc};
                            \draw
                            (q0) edge[above] node{$(0 \cup 10)^\ast(1 \cup 111^\ast\epsilon \cup \epsilon)$} (qend);
                        \end{tikzpicture}
                        \caption{GNFA - removed state A}
                    \end{figure}
              \item $\{w|\text{the length of }w\text{ is at most }5\}$
                    \begin{align*}
                        (0+1)?(0+1)?(0+1)?(0+1)?(0+1)?
                    \end{align*}
              \item $\{w|w~ \text{is any string except }11\text{ and }111\}$
                    \begin{align*}
                        (\Sigma)^* - 11 - 111
                    \end{align*}
              \item $\{w|\text{ every odd position of }w\text{ is a }1\}$
                    \begin{align*}
                        1((\Sigma)1)^*
                    \end{align*}
              \item $\{w|w~ \text{contains at least two }0\text{'s and at most one }1\}$
                    \begin{align*}
                        0^*0^*1?0^*
                    \end{align*}
              \item $\{\epsilon,0\}$
                    \begin{align*}
                        0?
                    \end{align*}
              \item $\{w|w~ \text{contains an even number of }0\text{'s, or contains exactly two }1\text{'s}\}$
                    \begin{align*}
                        (1^*01^*01^*)^* \cup 0^*10^*10^*
                    \end{align*}
              \item The empty set
                    \begin{align*}
                        1^*\emptyset
                    \end{align*}
              \item All strings except the empty string
                    \begin{align*}
                        \Sigma\Sigma^+
                    \end{align*}
          \end{enumerate}
\end{enumerate}
