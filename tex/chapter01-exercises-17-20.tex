\begin{enumerate}
    \item[1.7]
          Give state diagrams of NFAs with the specified number of states recognizing each of the following languages. In all parts, the alphabet is {0,1}.
          \begin{enumerate}
              \item The language $\{w|w~ \text{ends with }00\}$ with three states
                    \begin{figure}[H]
                        \centering
                        \begin{tikzpicture}
                            \node[state, initial] (q1) {};
                            \node[state, right of=q1] (q2) {};
                            \node[state, accepting, right of=q2] (q3) {};
                            \draw
                            (q1) edge[above] node{$0$} (q2)
                            (q2) edge[above] node{$0$} (q3)
                            (q1) edge[loop above] node{$0,1$} (q1);
                        \end{tikzpicture}
                    \end{figure}
              \item The language of Exercise 1.6c with five states
                    \begin{figure}[H]
                        \centering
                        \begin{tikzpicture}
                            \node[state, initial] (q1) {};
                            \node[state, right of=q1] (q2) {};
                            \node[state, right of=q2] (q3) {};
                            \node[state, right of=q3] (q4) {};
                            \node[state, accepting, right of=q4] (q5) {};
                            \draw
                            (q1) edge[loop above] node{$1$} (q1)
                            (q1) edge[above] node{$0$} (q2)
                            (q2) edge[loop above] node{$0$} (q2)
                            (q2) edge[above] node{$1$} (q3)
                            (q3) edge[above] node{$0$} (q4)
                            (q4) edge[above] node{$1$} (q5)
                            (q5) edge[loop above] node{$0,1$} (q5)
                            (q3) edge[bend left, below] node{$1$} (q1)
                            (q4) edge[bend right, above] node{$0$} (q2);
                        \end{tikzpicture}
                    \end{figure}
              \item The language of Exercise 1.6l with six states
                    \begin{figure}[H]
                        \centering
                        \begin{tikzpicture}
                            \node[state, accepting, initial] (q1) {};
                            \node[state, above right of=q1] (q2) {};
                            \node[state, accepting, below right of=q1] (q3) {};
                            \node[state, right of=q3] (q4) {};
                            \node[state, right of=q2] (q5) {};
                            \node[state, accepting, right of=q5] (q6) {};
                            \draw
                            (q1) edge[right] node{$\epsilon$} (q2)
                            (q1) edge[above] node{$1$} (q3)
                            (q3) edge[above] node{$1$} (q4)
                            (q1) edge[loop above] node{$0$} (q1)
                            (q3) edge[loop above] node{$0$} (q3)
                            (q4) edge[loop above] node{$0$} (q4)
                            (q2) edge[loop above] node{$0,1$} (q2)
                            (q5) edge[loop above] node{$1$} (q5)
                            (q6) edge[loop above] node{$0,1$} (q6)
                            (q2) edge[right, below] node{$0$} (q5)
                            (q5) edge[right, below] node{$0$} (q6);
                        \end{tikzpicture}
                    \end{figure}
              \item The language $\{0\}$ with two states
                    \begin{figure}[H]
                        \centering
                        \begin{tikzpicture}
                            \node[state, initial] (q1) {};
                            \node[state, accepting, right of=q1] (q2) {};
                            \draw
                            (q1) edge[above] node{$0$} (q2);
                        \end{tikzpicture}
                    \end{figure}
              \item The language $0^\ast1^\ast0+$ with three states
                    \begin{figure}[H]
                        \centering
                        \begin{tikzpicture}
                            \node[state, initial] (q1) {};
                            \node[state, right of=q1] (q2) {};
                            \node[state, accepting, right of=q2] (q3) {};
                            \draw
                            (q1) edge[above] node{$\epsilon$} (q2)
                            (q2) edge[above] node{$0$} (q3)
                            (q1) edge[loop above] node{$0$} (q1)
                            (q2) edge[loop above] node{$1$} (q2)
                            (q3) edge[loop above] node{$0$} (q3);
                        \end{tikzpicture}
                    \end{figure}
              \item The language $1^\ast(001+)^\ast$ with three states
                    \begin{figure}[H]
                        \centering
                        \begin{tikzpicture}
                            \node[state, accepting, initial] (q1) {};
                            \node[state, right of=q1] (q2) {};
                            \node[state, below right of=q1] (q3) {};
                            \draw
                            (q1) edge[above] node{$0$} (q2)
                            (q2) edge[right] node{$0$} (q3)
                            (q3) edge[left] node{$1$} (q1)
                            (q1) edge[loop above] node{$1$} (q1);
                        \end{tikzpicture}
                    \end{figure}
              \item The language ${\epsilon}$ with one state
                    \begin{figure}[H]
                        \centering
                        \begin{tikzpicture}
                            \node[state, initial, accepting] (q1) {};
                        \end{tikzpicture}
                    \end{figure}
              \item The language $0^\ast$ with one state
                    \begin{figure}[H]
                        \centering
                        \begin{tikzpicture}
                            \node[state, initial, accepting] (q1) {};
                            \draw
                            (q1) edge[loop above] node{$0$} (q1);
                        \end{tikzpicture}
                    \end{figure}
          \end{enumerate}
    \item[1.8]
          Use the construction in the proof of Theorem 1.45 to give the state diagrams of NFAs recognizing the union of the languages described in:
          \begin{enumerate}
              \item  Exercises 1.6a and 1.6b
                    \begin{figure}[H]
                        \centering
                        \begin{tikzpicture}
                            \node[state, initial] (q0) {};
                            \node[state, above right of=q0] (q1) {};
                            \node[state, right of=q1] (q2) {};
                            \node[state, accepting, right of=q2] (q3) {};
                            \node[state, right of=q3] (q4) {};
                            \node[state, below right of=q0] (bq1) {};
                            \node[state, right of=bq1] (bq2) {};
                            \node[state, right of=bq2] (bq3) {};
                            \node[state, accepting, right of=bq3] (bq4) {};
                            \draw
                            (q0) edge[above, blue] node{$\epsilon$} (q1)
                            (q0) edge[above, blue] node{$\epsilon$} (bq1)
                            (q1) edge[above] node{$1$} (q2)
                            (q2) edge[loop above] node{$1$} (q2)
                            (q2) edge[bend left, above] node{$0$} (q3)
                            (q3) edge[loop above] node{$0$} (q3)
                            (q3) edge[bend left, below] node{$1$} (q2)
                            (q1) edge[bend right, below] node{$0$} (q4)
                            (q4) edge[loop above] node{$0,1$} (q4)
                            (q1) edge[loop above] node{$0$} (q1)
                            (bq1) edge[above] node{$1$} (bq2)
                            (bq2) edge[loop above] node{$0$} (bq2)
                            (bq2) edge[above] node{$1$} (bq3)
                            (bq3) edge[loop above] node{$0$} (bq3)
                            (bq3) edge[above] node{$1$} (bq4)
                            (bq4) edge[loop above] node{$0,1$} (bq4);
                        \end{tikzpicture}
                    \end{figure}
              \item Exercises 1.6c and 1.6f
                    \begin{figure}[H]
                        \centering
                        \begin{tikzpicture}
                            \node[state, initial] (q0) {};
                            \node[state, above right of=q0] (q1) {};
                            \node[state, right of=q1] (q2) {};
                            \node[state, right of=q2] (q3) {};
                            \node[state, right of=q3] (q4) {};
                            \node[state, accepting, right of=q4] (q5) {};
                            \node[state, accepting, below right of=q0] (bq1) {};
                            \node[state, accepting, right of=bq1] (bq2) {};
                            \node[state, accepting, right of=bq2] (bq3) {};
                            \node[state, right of=bq3] (bq4) {};
                            \draw
                            (q0) edge[above, blue] node{$\epsilon$} (q1)
                            (q0) edge[above, blue] node{$\epsilon$} (bq1)
                            (q1) edge[loop above] node{$1$} (q1)
                            (q1) edge[above] node{$0$} (q2)
                            (q2) edge[loop above] node{$0$} (q2)
                            (q2) edge[above] node{$1$} (q3)
                            (q3) edge[above] node{$0$} (q4)
                            (q4) edge[above] node{$1$} (q5)
                            (q5) edge[loop above] node{$0,1$} (q5)
                            (q3) edge[bend left, below] node{$1$} (q1)
                            (q4) edge[bend right, above] node{$0$} (q2)
                            (bq1) edge[above] node{$1$} (bq2)
                            (bq1) edge[loop above] node{$0$} (bq1)
                            (bq2) edge[above] node{$1$} (bq3)
                            (bq2) edge[bend right, above] node{$0$} (bq1)
                            (bq3) edge[above] node{$0$} (bq4)
                            (bq3) edge[loop above] node{$1$} (bq3)
                            (bq4) edge[loop above] node{$0,1$} (bq4);
                        \end{tikzpicture}
                    \end{figure}
          \end{enumerate}
    \item[1.9]
          Use the construction in the proof of Theorem 1.47 to give the state diagrams of NFAs recognizing the concatenation of the languages described in
          \begin{enumerate}
              \item Exercises 1.6g and 1.6i.
                    \begin{figure}[H]
                        \centering
                        \begin{tikzpicture}
                            \node[state, accepting, initial] (q1) {};
                            \node[state, accepting, below of=q1] (q2) {};
                            \node[state, accepting, right of=q2] (q3) {};
                            \node[state, accepting, above of=q3] (q4) {};
                            \node[state, accepting, right of=q4] (q5) {};
                            \node[state, accepting, below of=q5] (q6) {};
                            \node[state, right of=q6] (q7) {};
                            \node[state, below of=q7] (bq1) {};
                            \node[state, above right of=bq1] (bq2) {};
                            \node[state, accepting, below right of=bq1] (bq3) {};
                            \node[state, accepting, right of=bq3] (bq5) {};
                            \draw
                            (q7) edge[loop above] node{$0,1$} (q7)
                            (q1) edge[left] node{$0,1$} (q2)
                            (q2) edge[above] node{$0,1$} (q3)
                            (q3) edge[left] node{$0,1$} (q4)
                            (q4) edge[above] node{$0,1$} (q5)
                            (q5) edge[left] node{$0,1$} (q6)
                            (q6) edge[above] node{$0,1$} (q7)
                            
                            (bq1) edge[above] node{$0$} (bq2)
                            (bq1) edge[above] node{$1$} (bq3)
                            (bq2) edge[loop above] node{$0,1$} (bq2)
                            (bq3) edge[bend left, above] node{$0,1$} (bq5)
                            (bq5) edge[bend left, above] node{$1$} (bq3)
                            (bq5) edge[bend right, above] node{$0$} (bq2)
                            
                            (q1) edge[left, below, blue] node{$\epsilon$} (bq1)
                            (q2) edge[left, below, blue] node{$\epsilon$} (bq1)
                            (q3) edge[left, below, blue] node{$\epsilon$} (bq1)
                            (q4) edge[left, below, blue] node{$\epsilon$} (bq1)
                            (q5) edge[left, below, blue] node{$\epsilon$} (bq1);
                        \end{tikzpicture}
                    \end{figure}
              \item Exercises 1.6b and 1.6m.
                    \begin{figure}[H]
                        \centering
                        \begin{tikzpicture}
                            \node[state, initial] (q1) {};
                            \node[state, right of=q1] (q2) {};
                            \node[state, right of=q2] (q3) {};
                            \node[state, accepting, right of=q3] (q4) {};
                            \node[state, right of=q4] (bq1) {};
                            \draw
                            (q1) edge[loop above] node{$0$} (q1)
                            (q1) edge[above] node{$1$} (q2)
                            (q2) edge[loop above] node{$0$} (q2)
                            (q2) edge[above] node{$1$} (q3)
                            (q3) edge[loop above] node{$0$} (q3)
                            (q3) edge[above] node{$1$} (q4)
                            (q4) edge[loop above] node{$0,1$} (q4)
                            (q4) edge[above, blue] node{$\epsilon$} (bq1)
                            (bq1) edge[loop right] node{$0,1$} (bq1);
                        \end{tikzpicture}
                    \end{figure}
          \end{enumerate}
          
    \item [1.10]
          Use the construction in the proof of Theorem 1.49 to give the state diagrams of NFAs recognizing the star of the languages described in 
          \begin{enumerate}
              \item Exercise 1.6b.
                    \begin{figure}[H]
                        \centering
                        \begin{tikzpicture}
                            \node[state, accepting, initial] (q0) {};
                            \node[state, right of=q0] (q1) {};
                            \node[state, right of=q1] (q2) {};
                            \node[state, right of=q2] (q3) {};
                            \node[state, accepting, right of=q3] (q4) {};
                            \draw
                            (q0) edge[above] node{$\epsilon$} (q1)
                            
                            (q1) edge[loop above] node{$0$} (q1)
                            (q1) edge[above] node{$1$} (q2)
                            (q2) edge[loop above] node{$0$} (q2)
                            (q2) edge[above] node{$1$} (q3)
                            (q3) edge[loop above] node{$0$} (q3)
                            (q3) edge[above] node{$1$} (q4)
                            (q4) edge[loop above] node{$0,1$} (q4)
                            
                            (q4) edge[bend left, below] node{$\epsilon$} (q1);
                        \end{tikzpicture}
                    \end{figure}
              \item Exercise 1.6j.
                    \begin{figure}[H]
                        \centering
                        \begin{tikzpicture}
                            \node[state, initial] (q0) {};
                            \node[state, right of=q0] (q1) {};
                            \node[state, above right of=q1] (q2) {};
                            \node[state, below right of=q1] (q3) {};
                            \node[state, below right of=q2] (q4) {};
                            \node[state, accepting, above right of=q2] (q5) {};
                            \node[state, below right of=q3] (q8) {};
                            \node[state, accepting, right of=q4] (q6) {};
                            \draw
                            (q0) edge[above] node{$\epsilon$} (q1)
                            (q1) edge[above] node{$0$} (q2)
                            (q1) edge[above] node{$1$} (q3)
                            (q2) edge[above] node{$0$} (q5)
                            (q2) edge[above] node{$1$} (q4)
                            (q3) edge[above] node{$0$} (q4)
                            (q3) edge[above] node{$1$} (q8)
                            (q4) edge[above] node{$0$} (q6)
                            (q4) edge[right] node{$1$} (q8)
                            (q5) edge[above] node{$1$} (q6)
                            (q5) edge[loop right] node{$0$} (q5)
                            (q6) edge[loop above] node{$0$} (q6)
                            (q8) edge[loop right] node{$0,1$} (q8)
                            (q6) edge[above] node{$1$} (q8)
                            
                            (q5) edge[bend left, below] node{$\epsilon$} (q1)
                            (q6) edge[bend left, below] node{$\epsilon$} (q1);
                        \end{tikzpicture}
                    \end{figure}
              \item Exercise 1.6m.
                    \begin{figure}[H]
                        \centering
                        \begin{tikzpicture}
                            \node[state, initial, accepting] (q0) {};
                            \node[state, right of=q0] (q1) {};
                            \draw
                            (q0) edge[above] node{$\epsilon$} (q1)
                            (q1) edge[loop right] node{$0,1$} (q1);
                        \end{tikzpicture}
                    \end{figure}
          \end{enumerate}
          
    \item [1.11]
          Prove that every NFA can be converted to an equivalent one that has a single accept state.
          
          It is enough to show that every NFA can be converted to an equivalent one that has a single accept state and no transitions into the accept state. Let $N = (Q, \Sigma, \delta, q_0, F)$ be an NFA. We construct an NFA $N' = (Q \cup \{q_f\}, \Sigma, \delta', q_0, \{q_f\})$ where $\delta'$ is the same as $\delta$ with additional transitions: for each $q \in F$ $\delta'(q, \epsilon) = \{q_f\}$. It is clear that $L(N) = L(N')$ and that $N'$ has a single accept state. Therefore, every NFA can be converted to an equivalent one that has a single accept state.
          
    \item [1.12]
          
          Let \[D = \{w|w~ \text{contains an even number of }a\text{’s and an odd number of }b\text{’s}\]
          \[ \text{and does not contain the substring }ab\}\] Give a DFA with five states that recognizes $D$ and a regular expression that generates $D$. (Suggestion: Describe $D$ more simply.)
          
          $$D = \{w|w~ \text{contains odd number of }b\text{'s followed by even number of }a\text{'s} \}$$
          
          \begin{figure}[H]
              \centering
              \begin{tikzpicture}
                  \node[state, initial] (q0) {};
                  \node[state, accepting, right of=q0] (q1) {};
                  \node[state, right of=q1] (q2) {};
                  \node[state, right of=q2] (q3) {};
                  \node[state, above of=q1] (q5) {};
                  \draw
                  (q0) edge[above] node{$b$} (q1)
                  (q0) edge[bend right, below] node{$a$} (q2)
                  (q1) edge[bend left, left] node{$b$} (q5)
                  (q5) edge[bend left, left] node{$b$} (q1)
                  (q1) edge[bend left, below] node{$a$} (q2)
                  (q2) edge[bend left, above] node{$a$} (q1)
                  (q2) edge[below] node{$b$} (q3)
                  (q5) edge[below] node{$a$} (q3)
                  (q3) edge[loop above] node{$a,b$} (q3);
              \end{tikzpicture}
          \end{figure}
          
          The regular expression that generates $D$ is $b(bb)^{\ast}(aa)^{\ast}$.
          
    \item [1.13]

    Let $F$ be the language of all strings over $\{0,1\}$ that do not contain a pair of $1$s that are separated by an odd number of symbols. Give the state diagram of a DFA with five states that recognizes $F$. (You may find it helpful first to find a 4-state NFA for the complement of $F$.)

    $\overline{F} = \{w|w~ \text{contains a pair of }1\text{'s that are separated by an odd number of symbols}\}$

    \begin{figure}[H]
        \centering
        \begin{tikzpicture}
            \node[state, initial] (q0) {};
            \node[state, right of=q0] (q1) {};
            \node[state, right of=q1] (q2) {};
            \node[state, accepting, right of=q2] (q3) {};
            \node[state, right of=q3] (q4) {};
            \draw
            (q0) edge[above] node{$0$} (q1)
            (q0) edge[bend right, below] node{$1$} (q2)
            (q1) edge[above] node{$0$} (q3)
            (q1) edge[bend right, below] node{$1$} (q4)
            (q2) edge[above] node{$0$} (q4)
            (q2) edge[bend right, above] node{$1$} (q3)
            (q3) edge[loop above] node{$0$} (q3)
            (q3) edge[bend right, above] node{$1$} (q4)
            (q4) edge[loop above] node{$0,1$} (q4);
        \end{tikzpicture}
    \end{figure}
\end{enumerate}
