\begin{enumerate}

    \item [1.36]
          Let $B_n = \{a^k |  ~k~\text{is a multiple of} ~n\}$. Show that for each $n \ge 1$, the language $B_n$ is regular.

          Lets construct NFA accepting

          For $n = 2$:
          \begin{figure}[H]
              \centering
              \begin{tikzpicture}[
                  ]
                  \node[state, initial, accepting] (q0) {$0$};
                  \node[state, right of=q0] (q1) {$1$};
                  \node[state, right of=q1] (q2) {$2$};
                  \draw
                  (q0) edge[above] node{$a$} (q1)
                  (q1) edge[above] node{$a$} (q2)
                  (q2) edge[bend right, above] node{$\epsilon$} (q0);
              \end{tikzpicture}
              \caption{NFA recognicing $B_2$}
          \end{figure}

          \begin{figure}[H]
              \centering
              \begin{tikzpicture}[
                  ]
                  \node[state, initial, accepting] (q0) {$0$};
                  \node[state, right of=q0] (q1) {$1$};
                  \node[state, right of=q1] (q2) {$2$};
                  \node[state, right of=q2] (qe1) {$n-1$};
                  \node[state, right of=qe1] (qe) {$n$};
                  \draw
                  (q0) edge[above] node{$a$} (q1)
                  (q1) edge[above] node{$a$} (q2)
                  (q2) edge[above, dashed] node{$a$} (qe1)
                  (qe1) edge[above] node{$a$} (qe)
                  (qe) edge[bend right, above] node{$\epsilon$} (q0);
              \end{tikzpicture}
              \caption{NFA recognicing $B_n$}
          \end{figure}

    \item [1.37]
          Let $C_n = \{x~ |~ x~ \text{is a binary number that is a multiple of} ~n\}$. Show that for each $n \ge 1$, the language $C_n$ is regular.

          Lets build $DFA_C$ recognizing $C_n$:

          $DFA_C = (Q, \Sigma, \delta, q_0, F)$
          \begin{itemize}
              \item $Q = \{0', 1', 2', \ldots, (n-1)'\}$
              \item $\Sigma = \{0, 1\}$
              \item $q_0 = 0'$
              \item $F = \{0'\}$
              \item $\delta$ is defined as follows:
                    \begin{align*}
                        \delta(q', 0) & = 2q \Mod{n}                                   \\
                        \delta(q', 1) & = (2q + 1) \Mod{n} = (2q \Mod{n} + 1 ) \Mod{n}
                    \end{align*}
                    where q is the integer part of $q'$.
          \end{itemize}
          We build DFA recognizing $C_n$ so $C_n$ is regular.

          Idea

          $i\text{-th}$ state represents integers which modulo from dividing by $n$ is $i$, because $n$ is multiple of $n$.

          \begin{figure}[H]
              \centering
              \begin{tikzpicture}[
                  ]
                  \node[state, initial, accepting] (q0) {$0'$};
                  \node[state, right of=q0] (q1) {$1'$};
                  \node[state, right of=q1] (q2) {$2'$};
                  \node[state, right of=q2] (q3) {$3'$};
                  \draw
                  (q0) edge[loop above] node{$0$} (q0)
                  (q0) edge[bend left, above] node{$1$} (q1)
                  (q1) edge[bend left, below] node{$0$} (q2)
                  (q1) edge[bend left, above] node{$1$} (q3)
                  (q2) edge[bend left, below] node{$0$} (q0)
                  (q2) edge[bend left, above] node{$1$} (q1)
                  (q3) edge[bend left, above] node{$0$} (q2)
                  (q3) edge[loop above] node{$1$} (q3);
              \end{tikzpicture}
              \caption{$DFA_C$ for $n = 4$}
          \end{figure}

          \begin{table}[H]
              \centering
              \begin{tabular}{|r|r|r|r|}
                  \hline
                  $0'$: mod 4 = 0      & $1'$: mod  4 = 1     & $2'$: mod 4 = 2       & $3'$: mod 4 = 3       \\
                  \hline
                  $0 = (0)_2$    & $1 = (1)_2$    & $2 = (10)_2$    & $3 = (11)_2$    \\
                  $4 = (100)_2$  & $5 = (101)_2$  & $6 = (110)_2$   & $7 = (111)_2$   \\
                  $8 = (1000)_2$ & $9 = (1001)_2$ & $10 = (1010)_2$ & $11 = (1011)_2$ \\
                  \ldots         & \ldots         & \ldots          & \ldots          \\
                  \hline
              \end{tabular}
          \end{table}

    \item [1.38]

    An all-NFA $M$ is a 5-tuple $(Q,\Sigma,\delta,q_0,F)$ that accepts $x \in \Sigma^\ast$ if every possible state that $M$ could be in after reading input $x$ is a state from $F$. Note, in contrast, that an ordinary NFA accepts a string if some state among these possible states is an accept state. Prove that all-NFAs recognize the class of regular languages.

    In procedure to change NFA to DFA just change step which marks state as accepting when state of DFA corresponds to states of NFA with at least one is acccepting to accepting only when all states are accepting.

    Now we have DFA avvepting the same language as all-NFA $M$ so language is regular.
    \item [1.39]
    \item [1.40]
\end{enumerate}
