\begin{enumerate}

    \item [1.36]
          Let $B_n = \{a^k |  ~k~\text{is a multiple of} ~n\}$. Show that for each $n \ge 1$, the language $B_n$ is regular.

          Lets construct NFA accepting

          For $n = 2$:
          \begin{figure}[H]
              \centering
              \begin{tikzpicture}[
                  ]
                  \node[state, initial, accepting] (q0) {$0$};
                  \node[state, right of=q0] (q1) {$1$};
                  \node[state, right of=q1] (q2) {$2$};
                  \draw
                  (q0) edge[above] node{$a$} (q1)
                  (q1) edge[above] node{$a$} (q2)
                  (q2) edge[bend right, above] node{$\epsilon$} (q0);
              \end{tikzpicture}
              \caption{NFA recognicing $B_2$}
          \end{figure}

          \begin{figure}[H]
              \centering
              \begin{tikzpicture}[
                  ]
                  \node[state, initial, accepting] (q0) {$0$};
                  \node[state, right of=q0] (q1) {$1$};
                  \node[state, right of=q1] (q2) {$2$};
                  \node[state, right of=q2] (qe1) {$n-1$};
                  \node[state, right of=qe1] (qe) {$n$};
                  \draw
                  (q0) edge[above] node{$a$} (q1)
                  (q1) edge[above] node{$a$} (q2)
                  (q2) edge[above, dashed] node{$a$} (qe1)
                  (qe1) edge[above] node{$a$} (qe)
                  (qe) edge[bend right, above] node{$\epsilon$} (q0);
              \end{tikzpicture}
              \caption{NFA recognicing $B_n$}
          \end{figure}

    \item [1.37]
          Let $C_n = \{x~ |~ x~ \text{is a binary number that is a multiple of} ~n\}$. Show that for each $n \ge 1$, the language $C_n$ is regular.

          Lets build $DFA_C$ recognizing $C_n$:

          $DFA_C = (Q, \Sigma, \delta, q_0, F)$
          \begin{itemize}
              \item $Q = \{0', 1', 2', \ldots, (n-1)'\}$
              \item $\Sigma = \{0, 1\}$
              \item $q_0 = 0'$
              \item $F = \{0'\}$
              \item $\delta$ is defined as follows:
                    \begin{align*}
                        \delta(q', 0) & = 2q \Mod{n}                                   \\
                        \delta(q', 1) & = (2q + 1) \Mod{n} = (2q \Mod{n} + 1 ) \Mod{n}
                    \end{align*}
                    where q is the integer part of $q'$.
          \end{itemize}
          We build DFA recognizing $C_n$ so $C_n$ is regular.

          Idea

          $i\text{-th}$ state represents integers which modulo from dividing by $n$ is $i$, because $n$ is multiple of $n$.

          \begin{figure}[H]
              \centering
              \begin{tikzpicture}[
                  ]
                  \node[state, initial, accepting] (q0) {$0'$};
                  \node[state, right of=q0] (q1) {$1'$};
                  \node[state, right of=q1] (q2) {$2'$};
                  \node[state, right of=q2] (q3) {$3'$};
                  \draw
                  (q0) edge[loop above] node{$0$} (q0)
                  (q0) edge[bend left, above] node{$1$} (q1)
                  (q1) edge[bend left, below] node{$0$} (q2)
                  (q1) edge[bend left, above] node{$1$} (q3)
                  (q2) edge[bend left, below] node{$0$} (q0)
                  (q2) edge[bend left, above] node{$1$} (q1)
                  (q3) edge[bend left, above] node{$0$} (q2)
                  (q3) edge[loop above] node{$1$} (q3);
              \end{tikzpicture}
              \caption{$DFA_C$ for $n = 4$}
          \end{figure}

          \begin{table}[H]
              \centering
              \begin{tabular}{|r|r|r|r|}
                  \hline
                  $0'$: mod 4 = 0 & $1'$: mod  4 = 1 & $2'$: mod 4 = 2 & $3'$: mod 4 = 3 \\
                  \hline
                  $0 = (0)_2$     & $1 = (1)_2$      & $2 = (10)_2$    & $3 = (11)_2$    \\
                  $4 = (100)_2$   & $5 = (101)_2$    & $6 = (110)_2$   & $7 = (111)_2$   \\
                  $8 = (1000)_2$  & $9 = (1001)_2$   & $10 = (1010)_2$ & $11 = (1011)_2$ \\
                  \ldots          & \ldots           & \ldots          & \ldots          \\
                  \hline
              \end{tabular}
          \end{table}

    \item [1.38]

          An all-NFA $M$ is a 5-tuple $(Q,\Sigma,\delta,q_0,F)$ that accepts $x \in \Sigma^\ast$ if every possible state that $M$ could be in after reading input $x$ is a state from $F$. Note, in contrast, that an ordinary NFA accepts a string if some state among these possible states is an accept state. Prove that all-NFAs recognize the class of regular languages.

          In procedure to change NFA to DFA just change step which marks state as accepting when state of DFA corresponds to states of NFA with at least one is acccepting to accepting only when all states are accepting.

          Now we have DFA accepting the same language as all-NFA $M$ so language is regular.
    \item [1.39]
          The construction in Theorem 1.54 shows that every GNFA is equivalent to a GNFA with only two states. We can show that an opposite phenomenon occurs for DFAs. Prove that for every $k > 1$, a language $A_k \subseteq \{0,1\}^\ast$ exists that is recognized by a DFA with $k$ states but not by one with only $k-1$ states.

          TODO: check solution

          Let $A_k = \{0^\ast10^\ast \ldots 0^\ast10^\ast ~|~ \text{where it is exactly}~k-1~\text{ones}\}$.

          We cannot build DFA with $k-1$ states recognizing $A_k$ because we need to remember how many ones we have seen so far. We can't do this with $k-1$ states because we can't remember how many ones we have seen so far. (not sure if this is enough to prove it)

    \item [1.40]
          Recall that string $x$ is a \textbf{prefix} of string $y$ if a string $z$ exists where $xz = y$, and that $x$ is a \textbf{proper prefix} of $y$ if in addition $x \neq y$. In each of the following parts, we define an operation on a language $A$. Show that the class of regular languages is closed under that operation.
          \begin{itemize}
              \item $L = \text{NOPREFIX}(A)=\{w \in A~|~\text{no proper prefix of } w~ \text{is a member of}~ A\}$.

                    Let be DFA $M = (Q, \Sigma, \delta, q_0, F)$ recognizes A (as A is regular language).

                    We need to construct $L$ which contains strings so that DFA $M$ will go to accepting state only at the end of processing string. In other words for all $w = w_1w_2\ldots w_n \in A$, $\delta(q_{i-1}, w_i) = q_{i}$ and $q_i \in F$ only for $i = n$.

                    DFA $M_{NOPREFIX} = (Q \cup {q_f}, \Sigma, \delta_{NOPREFIX}, q_0, F)$

                    where $\delta_{NOPREFIX}(q, a) = \delta(q, a)$ for $q \in Q$ and $\delta_{NOPREFIX}(q, a) = q_f$ for $q \in F$.


              \item $\text{NOEXTEND}(A)=\{w \in A~|~w~ \text{is not the proper prefix of any string in}~ A\}$.

                    During processing string $w$ DFA $M$ can't go from accepting state to another accepting state by any possible way.

                    DFA $M_{NOEXTEND} = (Q, \Sigma, \delta, q_0, F_{NOEXTEND})$

                    where $F_{NOEXTEND} = F - F'$

                    where $F'$ is set of states which can be reached from accepting state by any possible way.

                    To build $F'$ lets go from all accepting states and mark all states which can be reached from them. Then $F'$ is set of all marked states. We can stop after $\overline{Q}$ steps where $\overline{Q}$ is number of states in DFA $M$. Each step we go every $a \in \Sigma$ from every accepting state or state reached from previous step.
          \end{itemize}
\end{enumerate}
