\begin{enumerate}

    \item[1.1]
          The following are the state diagrams of two DFAs, $M_1$ and $M_2$. Answer the following questions about each of these machines.
          
          \begin{figure}[H]
              \centering
              \begin{tikzpicture}
                  \node[state, initial] (q1) {$q_1$};
                  \node[state, accepting, right of=q1] (q2) {$q_2$};
                  \node[state] at (1.5, -2) (q3) {$q_3$};
                  \draw (q1) edge[loop above] node{$b$} (q1)
                  (q1) edge[bend left, above] node{$a$} (q2)
                  (q2) edge[bend left, below] node{$a,b$} (q3)
                  (q3) edge[bend left, above] node{$a$} (q2)
                  (q3) edge[bend left, below] node{$b$} (q1);
              \end{tikzpicture}
              \begin{tikzpicture}
                  \node[state, initial] (q1) {$q_1$};
                  \node[state, right of=q1] (q2) {$q_2$};
                  \node[state, below of=q1] (q3) {$q_3$};
                  \node[state, accepting, below of=q2] (q4) {$q_4$};
                  \draw (q1) edge[loop above] node{$a$} (q1)
                  (q1) edge[bend left, above] node{$b$} (q2)
                  (q2) edge[bend left, below] node{$a$} (q3)
                  (q2) edge[bend left, right] node{$b$} (q4)
                  (q3) edge[bend left, above] node{$a$} (q2)
                  (q3) edge[bend left, right] node{$b$} (q1)
                  (q4) edge[bend left, below] node{$a$} (q3)
                  (q4) edge[loop right] node{$b$} (q4);
              \end{tikzpicture}
              \caption{$M_1$ and $M_2$}
          \end{figure}
          
          \begin{enumerate}
              \item What is the start state?
                    
                    $M_1: q_1$
                    
                    $M_2: q_1$
                    
              \item What is the set of accept states?
                    
                    $M_1: \{q_2\}$
                    
                    $M_2: \{q_4\}$
              \item What sequence of states does the machine go through on input $aabb$?
                    
                    $M_1: q_1 \rightarrow q_2 \rightarrow q_3 \rightarrow q_1 \rightarrow q_1$
                    
                    $M_2: q_1 \rightarrow q_2 \rightarrow q_4 \rightarrow q_4 \rightarrow q_4$
              \item Does the machine accept the string $aabb$?
                    
                    $M_1:$ No
                    
                    $M_2:$ Yes
              \item Does the machine accept the string $\epsilon$?
                    
                    $M_1:$ No
                    
                    $M_2:$ No
          \end{enumerate}
          
    \item[1.2]
    
          Give the formal description of the machines $M_1$ and $M_2$ pictured in Exercise 1.1.
          
          $M_1 = (Q, \Sigma_1, \delta_1, q_1, \{q_2\})$
          
          $Q = \{q_1, q_2, q_3\}$
          
          $\Sigma_1 = \{a, b\}$
          
          $\delta_1 = \{((q_1, a), q_2), ((q_1, b), q_1), ((q_2, a), q_3), ((q_2, b), q_3), ((q_3, a), q_2), ((q_3, b), q_1)\}$\\
          
          $M_2 = (Q, \Sigma_2, \delta_2, q_1, \{q_4\})$
          
          $Q = \{q_1, q_2, q_3, q_4\}$
          
          $\Sigma_2 = \{a, b\}$
          
          $\delta_2 = \{((q_1, a), q_1), ((q_1, b), q_2), ((q_2, a), q_3), ((q_2, b), q_4),$\\
          .\,\,\,\qquad$((q_3, a), q_2), ((q_3, b), q_1), ((q_4, a), q_3), ((q_4, b), q_4)\}$
          
    \item[1.3]
    
          The formal description of a DFA $M$ is $\{q_1,q_2,q_3,q_4,q_5\},\{u,d\},\sigma,q_3,\{q_3\}$, where $\sigma$ is given by the following table. Give the state diagram of this machine.
          
          \begin{center}


              \begin{tabular}{ c | c c }
                        & $u$   & $d$   \\
                  \hline
                  $q_1$ & $q_1$ & $q_2$ \\
                  $q_2$ & $q_1$ & $q_3$ \\
                  $q_3$ & $q_2$ & $q_4$ \\
                  $q_4$ & $q_3$ & $q_5$ \\
                  $q_5$ & $q_4$ & $q_5$ \\
              \end{tabular}
          \end{center}
          
          \begin{figure}[H]
              \centering
              \begin{tikzpicture}
                  \node[state] (q1) {$q_1$};
                  \node[state, below of=q1] (q2) {$q_2$};
                  \node[state, accepting, initial, below of=q2] (q3) {$q_3$};
                  \node[state, right of=q3] (q4) {$q_4$};
                  \node[state, right of=q4] (q5) {$q_5$};
                  \draw (q1) edge[loop above] node{$u$} (q1)
                  (q1) edge[bend left, left] node{$d$} (q2)
                  (q2) edge[bend left, right] node{$u$} (q1)
                  (q2) edge[bend left, left] node{$d$} (q3)
                  (q3) edge[bend left, right] node{$u$} (q2)
                  (q3) edge[bend left, above] node{$d$} (q4)
                  (q4) edge[bend left, above] node{$u$} (q3)
                  (q4) edge[bend left, above] node{$d$} (q5)
                  (q5) edge[loop above] node{$d$} (q5)
                  (q5) edge[bend left, above] node{$u$} (q4);
              \end{tikzpicture}
          \end{figure}
          
    \item[1.4]
          Each of the following languages is the intersection of two simpler languages. In each part, construct DFAs for the simpler languages,then combine them using the construction discussed in footnote 3 (page 46) to give the state diagram of a DFA for the language given. In all parts, $\Sigma=\{a,b\}$.
          \begin{enumerate}
              \item $\{w|w~\text{has at least three }a\text{’s and at least two }b\text{’s}\}$
              \item $\{w|w~\text{has exactly two }a\text{’s and at least two }b\text{’s}\}$
              \item $\{w|w~\text{hasan evennumber of }a\text{’s and one or two }b\text{’s}\}$
              \item $\{w|w~\text{hasan evennumber of }a\text{’s and each }a\text{ is followed by at least one }b\}$
              \item $\{w|w~\text{starts with an }a\text{ and has at most one }b\}$
              \item $\{w|w~\text{has an odd number of }a\text{’s and ends with a }b\}$
              \item $\{w|w~\text{has even length and an odd number of }a\text{’s}\}$
          \end{enumerate}
          
    \item[1.5]
          Each of the following languages is the complement of a simpler language. In each part, construct a DFA for the simpler language, then use it to give the state diagram of a DFA for the language given. In all parts, $\Sigma=\{a,b\}$.
          \begin{enumerate}
              \item $A = \{w|w~\text{does not contain the substring }ab\}$
                    
                    $\overline{A} = \{w|w~\text{contains the substring }ab\}$
                    \begin{figure}[H]
                        \centering
                        \begin{tikzpicture}
                            \node[state, initial] (q1) {};
                            \node[state, right of=q1] (q2) {};
                            \node[state, accepting, right of=q2] (q3) {};
                            \draw (q1) edge[left, above] node{$a$} (q2)
                            (q1) edge[loop above] node{$b$} (q1)
                            (q2) edge[left, above] node{$b$} (q3)
                            (q2) edge[loop above] node{$a$} (q2)
                            (q3) edge[loop above] node{$a,b$} (q3);
                        \end{tikzpicture}
                        \begin{tikzpicture}
                            \node[state, accepting, initial] (q1) {};
                            \node[state, accepting, right of=q1] (q2) {};
                            \node[state, right of=q2] (q3) {};
                            \draw (q1) edge[left, above] node{$a$} (q2)
                            (q1) edge[loop above] node{$b$} (q1)
                            (q2) edge[left, above] node{$b$} (q3)
                            (q2) edge[loop above] node{$a$} (q2)
                            (q3) edge[loop above] node{$a,b$} (q3);
                        \end{tikzpicture}
                        \caption{$\overline{A}$ and $A$}
                    \end{figure}
              \item $B = \{w|w~\text{does notcontain the substring }baba\}$
                    
                    $\overline{B} = \{w|w~\text{contains the substring }baba\}$
                    \begin{figure}[H]
                        \centering
                        \begin{tikzpicture}
                            \node[state, initial] (q1) {};
                            \node[state, right of=q1] (q2) {};
                            \node[state, right of=q2] (q3) {};
                            \node[state, right of=q3] (q4) {};
                            \node[state, accepting, right of=q4] (q5) {};
                            \draw (q1) edge[left, above] node{$b$} (q2)
                            (q1) edge[loop above] node{$a$} (q1)
                            (q2) edge[left, above] node{$a$} (q3)
                            (q2) edge[loop above] node{$b$} (q2)
                            (q3) edge[left, above] node{$b$} (q4)
                            (q3) edge[bend left, below] node{$a$} (q1)
                            (q4) edge[bend right, above] node{$b$} (q2)
                            (q4) edge[left, above] node{$a$} (q5)
                            (q5) edge[loop above] node{$a,b$} (q5);
                        \end{tikzpicture}
                        \begin{tikzpicture}
                            \node[state, accepting, initial] (q1) {};
                            \node[state, accepting, right of=q1] (q2) {};
                            \node[state, accepting, right of=q2] (q3) {};
                            \node[state, accepting, right of=q3] (q4) {};
                            \node[state, right of=q4] (q5) {};
                            \draw (q1) edge[left, above] node{$b$} (q2)
                            (q1) edge[loop above] node{$a$} (q1)
                            (q2) edge[left, above] node{$a$} (q3)
                            (q2) edge[loop above] node{$b$} (q2)
                            (q3) edge[left, above] node{$b$} (q4)
                            (q3) edge[bend left, below] node{$a$} (q1)
                            (q4) edge[bend right, above] node{$b$} (q2)
                            (q4) edge[left, above] node{$a$} (q5)
                            (q5) edge[loop above] node{$a,b$} (q5);
                        \end{tikzpicture}
                        \caption{$\overline{B}$ and $B$}
                    \end{figure}
                    
              \item $C =\{w|w~\text{contains neither the substrings }ab\text{ nor }ba\}$
                    
                    $\overline{C} = \{w|w~\text{contains the substring }ab\text{ or }ba\}$
                    \begin{figure}[H]
                        \centering
                        \begin{tikzpicture}
                            \node[state, initial] (q1) {};
                            \node[state, above right of=q1] (q2) {};
                            \node[state, below right of=q1] (q3) {};
                            \node[state, accepting, right of=q1] (q4) {};
                            \draw (q1) edge[left, above] node{$a$} (q2)
                            (q1) edge[left, above] node{$b$} (q3)
                            (q2) edge[right] node{$b$} (q4)
                            (q2) edge[loop right] node{$a$} (q2)
                            (q3) edge[right] node{$a$} (q4)
                            (q3) edge[loop right] node{$b$} (q3)
                            (q4) edge[loop right] node{$a,b$} (q4);
                        \end{tikzpicture}
                        \begin{tikzpicture}
                            \node[state, accepting, initial] (q1) {};
                            \node[state, accepting, above right of=q1] (q2) {};
                            \node[state, accepting, below right of=q1] (q3) {};
                            \node[state, right of=q1] (q4) {};
                            \draw (q1) edge[left, above] node{$a$} (q2)
                            (q1) edge[left, above] node{$b$} (q3)
                            (q2) edge[right] node{$b$} (q4)
                            (q2) edge[loop right] node{$a$} (q2)
                            (q3) edge[right] node{$a$} (q4)
                            (q3) edge[loop right] node{$b$} (q3)
                            (q4) edge[loop right] node{$a,b$} (q4);
                        \end{tikzpicture}
                        \caption{$\overline{C}$ and $C$}
                    \end{figure}
              \item $D = \{w|w~\text{is any string not in }a^{\ast}b^{\ast}\}$
                    
                    $\overline{D} = \{w|w~\text{is any string in }a^{\ast}b^{\ast}\}$
                    \begin{figure}[H]
                        \centering
                        \begin{tikzpicture}
                            \node[state, accepting, initial] (q1) {};
                            \node[state, accepting, right of=q1] (q2) {};
                            \node[state, right of=q2] (q3) {};
                            \draw (q1) edge[loop above] node{$a$} (q1)
                            (q1) edge[above] node{$b$} (q2)
                            (q2) edge[loop above] node{$b$} (q2)
                            (q2) edge[above] node{$a$} (q3)
                            (q3) edge[loop above] node{$a,b$} (q3);
                        \end{tikzpicture}
                        \begin{tikzpicture}
                            \node[state, initial] (q1) {};
                            \node[state, right of=q1] (q2) {};
                            \node[state, accepting, right of=q2] (q3) {};
                            \draw (q1) edge[loop above] node{$a$} (q1)
                            (q1) edge[above] node{$b$} (q2)
                            (q2) edge[loop above] node{$b$} (q2)
                            (q2) edge[above] node{$a$} (q3)
                            (q3) edge[loop above] node{$a,b$} (q3);
                        \end{tikzpicture}
                        \caption{$\overline{D}$ and $D$}
                    \end{figure}
              \item $E = \{w|w~\text{is any string not in }(ab^+)^{\ast}\}$
                    
                    $\overline{E} = \{w|w~\text{is any string in }(ab^+)^{\ast}\}$
                    
                    \begin{itemize}
                        \item assuming $ab^+$ means $a$ followed by one or more $b$.
                              \begin{figure}[H]
                                  \centering
                                  \begin{tikzpicture}
                                      \node[state, accepting, initial] (q1) {};
                                      \node[state, accepting, right of=q1] (q2) {};
                                      \draw (q1) edge[bend left, above] node{$a$} (q2)
                                      (q2) edge[bend left, below] node{$a$} (q1)
                                      (q2) edge[loop above] node{$b$} (q2);
                                  \end{tikzpicture}
                                  \begin{tikzpicture}
                                      \node[state, initial] (q1) {};
                                      \draw (q1) edge[loop above] node{$a,b$} (q1);
                                  \end{tikzpicture}
                                  \caption{$\overline{E}$ and $E$}
                              \end{figure}
                        \item assuming $ab^+$ means $ab$ one or more times.
                              \begin{figure}[H]
                                  \centering
                                  \begin{tikzpicture}
                                      \node[state, accepting, initial] (q1) {};
                                      \node[state, right of=q1] (q2) {};
                                      \node[state, accepting, right of=q2] (q3) {};
                                      \node[state, right of=q3] (q4) {};
                                      \draw (q1) edge[bend left, above] node{$a$} (q2)
                                      (q2) edge[bend left, above] node{$b$} (q3)
                                      (q3) edge[bend left, below] node{$a$} (q2)
                                      (q3) edge[bend left, below] node{$b$} (q4)
                                      (q1) edge[bend right, below] node{$b$} (q4)
                                      (q4) edge[loop above] node{$a,b$} (q4);
                                  \end{tikzpicture}
                                  \begin{tikzpicture}
                                      \node[state, initial] (q1) {};
                                      \node[state, accepting, right of=q1] (q2) {};
                                      \node[state, right of=q2] (q3) {};
                                      \node[state, accepting, right of=q3] (q4) {};
                                      \draw (q1) edge[bend left, above] node{$a$} (q2)
                                      (q2) edge[bend left, above] node{$b$} (q3)
                                      (q3) edge[bend left, below] node{$a$} (q2)
                                      (q3) edge[bend left, below] node{$b$} (q4)
                                      (q1) edge[bend right, below] node{$b$} (q4)
                                      (q4) edge[loop above] node{$a,b$} (q4);
                                  \end{tikzpicture}
                                  \caption{$\overline{E}$ and $E$}
                              \end{figure}
                    \end{itemize}
              \item $F = \{w|w~\text{is any string not in }a^{\ast} \cup b^{\ast}\}$
                    
                    $\overline{F} = \{w|w~\text{is any string in }a^{\ast} \cup b^{\ast}\}$
                    \begin{figure}[H]
                        \centering
                        \begin{tikzpicture}
                            \node[state, accepting, initial] (q1) {};
                            \node[state, accepting, right of=q1] (q2) {};
                            \node[state, right of=q2] (q3) {};
                            \draw (q1) edge[loop above] node{$a$} (q1)
                            (q1) edge[above] node{$b$} (q2)
                            (q2) edge[loop above] node{$b$} (q2)
                            (q2) edge[above] node{$a$} (q3)
                            (q3) edge[loop above] node{$a,b$} (q3);
                        \end{tikzpicture}
                        \begin{tikzpicture}
                            \node[state, initial] (q1) {};
                            \node[state, right of=q1] (q2) {};
                            \node[state, accepting, right of=q2] (q3) {};
                            \draw (q1) edge[loop above] node{$a$} (q1)
                            (q1) edge[above] node{$b$} (q2)
                            (q2) edge[loop above] node{$b$} (q2)
                            (q2) edge[above] node{$a$} (q3)
                            (q3) edge[loop above] node{$a,b$} (q3);
                        \end{tikzpicture}
                        \caption{$\overline{F}$ and $F$}
                    \end{figure}
              \item $G= \{w|w~\text{is any string that doesn’t contain exactly two }a\text{’s}\}$
                    
                    $\overline{G} = \{w|w~\text{is any string that contains exactly two }a\text{’s}\}$
                    \begin{figure}[H]
                        \centering
                        \begin{tikzpicture}
                            \node[state, initial] (q1) {};
                            \node[state, right of=q1] (q2) {};
                            \node[state, accepting, right of=q2] (q3) {};
                            \node[state, right of=q3] (q4) {};
                            \draw (q1) edge[loop above] node{$b$} (q1)
                            (q1) edge[above] node{$a$} (q2)
                            (q2) edge[loop above] node{$b$} (q2)
                            (q2) edge[above] node{$a$} (q3)
                            (q3) edge[loop above] node{$b$} (q3)
                            (q3) edge[above] node{$a$} (q4)
                            (q4) edge[loop above] node{$a,b$} (q4);
                        \end{tikzpicture}
                        \begin{tikzpicture}
                            \node[state, accepting, initial] (q1) {};
                            \node[state, accepting, right of=q1] (q2) {};
                            \node[state, right of=q2] (q3) {};
                            \node[state, accepting, right of=q3] (q4) {};
                            \draw (q1) edge[loop above] node{$b$} (q1)
                            (q1) edge[above] node{$a$} (q2)
                            (q2) edge[loop above] node{$b$} (q2)
                            (q2) edge[above] node{$a$} (q3)
                            (q3) edge[loop above] node{$b$} (q3)
                            (q3) edge[above] node{$a$} (q4)
                            (q4) edge[loop above] node{$a,b$} (q4);
                        \end{tikzpicture}
                        \caption{$\overline{G}$ and $G$}
                    \end{figure}
              \item $H=\{w|w~\text{is any string except }a\text{ and }b\}$
                    
                    $\overline{H} = \{w|w~\text{is string }a\text{ or }b\}$
                    \begin{figure}[H]
                        \centering
                        \begin{tikzpicture}
                            \node[state, initial] (q1) {};
                            \node[state, accepting, right of=q1] (q2) {};
                            \node[state, right of=q2] (q3) {};
                            \draw
                            (q1) edge[above] node{$a,b$} (q2)
                            (q2) edge[above] node{$a,b$} (q3)
                            (q3) edge[loop above] node{$a,b$} (q3);
                        \end{tikzpicture}
                        \begin{tikzpicture}
                            \node[state, accepting, initial] (q1) {};
                            \node[state, right of=q1] (q2) {};
                            \node[state, accepting, right of=q2] (q3) {};
                            \draw
                            (q1) edge[above] node{$a,b$} (q2)
                            (q2) edge[above] node{$a,b$} (q3)
                            (q3) edge[loop above] node{$a,b$} (q3);
                        \end{tikzpicture}
                        \caption{$\overline{H}$ and $H$}
                    \end{figure}
          \end{enumerate}
          
    \item[1.6]
          Give state diagrams of DFAs recognizing the following languages. In all parts,the alphabet is $\{0,1\}$.
          \begin{enumerate}
              \item $\{w|w~ \text{begins with a }1\text{ and ends with a }0\}$
                    \begin{figure}[H]
                        \centering
                        \begin{tikzpicture}
                            \node[state, initial] (q1) {};
                            \node[state, right of=q1] (q2) {};
                            \node[state, accepting, right of=q2] (q3) {};
                            \node[state, right of=q3] (q4) {};
                            \draw
                            (q1) edge[above] node{$1$} (q2)
                            (q2) edge[loop above] node{$1$} (q2)
                            (q2) edge[bend left, above] node{$0$} (q3)
                            (q3) edge[loop above] node{$0$} (q3)
                            (q3) edge[bend left, below] node{$1$} (q2)
                            (q1) edge[bend right, below] node{$0$} (q4)
                            (q4) edge[loop above] node{$0,1$} (q4);
                        \end{tikzpicture}
                    \end{figure}
              \item $\{w|w~ \text{contains at least three }1\text{'s}\}$
                    \begin{figure}[H]
                        \centering
                        \begin{tikzpicture}
                            \node[state, initial] (q1) {};
                            \node[state, right of=q1] (q2) {};
                            \node[state, right of=q2] (q3) {};
                            \node[state, accepting, right of=q3] (q4) {};
                            \draw
                            (q1) edge[loop above] node{$0$} (q1)
                            (q1) edge[above] node{$1$} (q2)
                            (q2) edge[loop above] node{$0$} (q2)
                            (q2) edge[above] node{$1$} (q3)
                            (q3) edge[loop above] node{$0$} (q3)
                            (q3) edge[above] node{$1$} (q4)
                            (q4) edge[loop above] node{$0,1$} (q4);
                        \end{tikzpicture}
                    \end{figure}
              \item $\{w|w~ \text{contains the substring } 0101~ \text{(i.e., }w = x0101y\text{ for some }x\text{ and }y)\}$
                    \begin{figure}[H]
                        \centering
                        \begin{tikzpicture}
                            \node[state, initial] (q1) {};
                            \node[state, right of=q1] (q2) {};
                            \node[state, right of=q2] (q3) {};
                            \node[state, right of=q3] (q4) {};
                            \node[state, accepting, right of=q4] (q5) {};
                            \draw
                            (q1) edge[loop above] node{$1$} (q1)
                            (q1) edge[above] node{$0$} (q2)
                            (q2) edge[loop above] node{$0$} (q2)
                            (q2) edge[above] node{$1$} (q3)
                            (q3) edge[above] node{$0$} (q4)
                            (q4) edge[above] node{$1$} (q5)
                            (q5) edge[loop above] node{$0,1$} (q5)
                            (q3) edge[bend left, below] node{$1$} (q1)
                            (q4) edge[bend right, above] node{$0$} (q2);
                        \end{tikzpicture}
                    \end{figure}
              \item $\{w|w~ \text{has length at least }3\text{ and its third symbol is a }0\}$
                    \begin{figure}[H]
                        \centering
                        \begin{tikzpicture}
                            \node[state, initial] (q1) {};
                            \node[state, right of=q1] (q2) {};
                            \node[state, right of=q2] (q3) {};
                            \node[state, accepting, below right of=q3] (q4) {};
                            \node[state, above right of=q3] (q5) {};
                            \draw
                            (q1) edge[above] node{$0,1$} (q2)
                            (q2) edge[above] node{$0,1$} (q3)
                            (q3) edge[above] node{$0$} (q4)
                            (q3) edge[above] node{$1$} (q5)
                            (q4) edge[loop above] node{$0,1$} (q4)
                            (q5) edge[loop above] node{$0,1$} (q5);
                        \end{tikzpicture}
                    \end{figure}
                    
              \item $\{w|w~ \text{starts with }0\text{ and has odd length, or starts with }1\text{ and has even length}\}$
                    \begin{figure}[H]
                        \centering
                        \begin{tikzpicture}
                            \node[state, initial] (q1) {};
                            \node[state, accepting, above right of=q1] (q2) {};
                            \node[state, below right of=q1] (q3) {};
                            \node[state, right of=q2] (q4) {};
                            \node[state, accepting, right of=q3] (q5) {};
                            \draw
                            (q1) edge[above] node{$0$} (q2)
                            (q1) edge[above] node{$1$} (q3)
                            (q2) edge[bend left, above] node{$0,1$} (q4)
                            (q4) edge[bend left, below] node{$0,1$} (q2)
                            (q3) edge[bend left, above] node{$0,1$} (q5)
                            (q5) edge[bend left, above] node{$0,1$} (q3);
                        \end{tikzpicture}
                    \end{figure}
              \item $\{w|w~ \text{doesn't contain the substring }110\}$
                    \begin{figure}[H]
                        \centering
                        \begin{tikzpicture}
                            \node[state, accepting, initial] (q1) {};
                            \node[state, accepting, right of=q1] (q2) {};
                            \node[state, accepting, right of=q2] (q3) {};
                            \node[state, right of=q3] (q4) {};
                            \draw
                            (q1) edge[above] node{$1$} (q2)
                            (q1) edge[loop above] node{$0$} (q1)
                            (q2) edge[above] node{$1$} (q3)
                            (q2) edge[bend right, above] node{$0$} (q1)
                            (q3) edge[above] node{$0$} (q4)
                            (q3) edge[loop above] node{$1$} (q3)
                            (q4) edge[loop above] node{$0,1$} (q4);
                        \end{tikzpicture}
                    \end{figure}
              \item $\{w|\text{the length of }w\text{ is at most }5\}$
                    
                    \begin{figure}[H]
                        \centering
                        \begin{tikzpicture}
                            \node[state, accepting, initial] (q1) {};
                            \node[state, accepting, below of=q1] (q2) {};
                            \node[state, accepting, right of=q2] (q3) {};
                            \node[state, accepting, above of=q3] (q4) {};
                            \node[state, accepting, right of=q4] (q5) {};
                            \node[state, accepting, below of=q5] (q6) {};
                            \node[state, right of=q6] (q7) {};
                            \draw
                            (q7) edge[loop above] node{$0,1$} (q7)
                            (q1) edge[left] node{$0,1$} (q2)
                            (q2) edge[above] node{$0,1$} (q3)
                            (q3) edge[left] node{$0,1$} (q4)
                            (q4) edge[above] node{$0,1$} (q5)
                            (q5) edge[left] node{$0,1$} (q6)
                            (q6) edge[above] node{$0,1$} (q7);
                        \end{tikzpicture}
                    \end{figure}
                    
              \item $\{w|w~ \text{is any string except }11\text{ and }111\}$
                    \begin{figure}[H]
                        \centering
                        \begin{tikzpicture}
                            \node[state, accepting, initial] (q1) {};
                            \node[state, accepting, right of=q1] (q2) {};
                            \node[state, right of=q2] (q3) {};
                            \node[state, right of=q3] (q4) {};
                            \node[state, accepting, right of=q4] (q5) {};
                            \draw
                            (q1) edge[above] node{$1$} (q2)
                            (q1) edge[bend left, above] node{$0$} (q5)
                            (q2) edge[above] node{$1$} (q3)
                            (q2) edge[bend left, below ] node{$0$} (q5)
                            (q3) edge[above] node{$1$} (q4)
                            (q3) edge[bend left, below] node{$0$} (q5)
                            (q4) edge[above] node{$0,1$} (q5)
                            (q5) edge[loop above] node{$0,1$} (q5);
                        \end{tikzpicture}
                    \end{figure}
              \item $\{w|\text{ every odd position of }w\text{ is a }1\}$
                    \begin{figure}[H]
                        \centering
                        \begin{tikzpicture}
                            \node[state, initial] (q1) {};
                            \node[state, above right of=q1] (q2) {};
                            \node[state, accepting, below right of=q1] (q3) {};
                            \node[state, accepting, right of=q3] (q5) {};
                            \draw
                            (q1) edge[above] node{$0$} (q2)
                            (q1) edge[above] node{$1$} (q3)
                            (q2) edge[loop above] node{$0,1$} (q2)
                            (q3) edge[bend left, above] node{$0,1$} (q5)
                            (q5) edge[bend left, above] node{$1$} (q3)
                            (q5) edge[bend right, above] node{$0$} (q2);
                        \end{tikzpicture}
                    \end{figure}
              \item $\{w|w~ \text{contains at least two }0\text{'s and at most one }1\}$
                    \begin{figure}[H]
                        \centering
                        \begin{tikzpicture}
                            \node[state, initial] (q1) {};
                            \node[state, above right of=q1] (q2) {};
                            \node[state, below right of=q1] (q3) {};
                            \node[state, below right of=q2] (q4) {};
                            \node[state, accepting, above right of=q2] (q5) {};
                            \node[state, below right of=q3] (q8) {};
                            \node[state, accepting, right of=q4] (q6) {};
                            \draw
                            (q1) edge[above] node{$0$} (q2)
                            (q1) edge[above] node{$1$} (q3)
                            (q2) edge[above] node{$0$} (q5)
                            (q2) edge[above] node{$1$} (q4)
                            (q3) edge[above] node{$0$} (q4)
                            (q3) edge[above] node{$1$} (q8)
                            (q4) edge[above] node{$0$} (q6)
                            (q4) edge[right] node{$1$} (q8)
                            (q5) edge[above] node{$1$} (q6)
                            (q5) edge[loop right] node{$0$} (q5)
                            (q6) edge[loop above] node{$0$} (q6)
                            (q8) edge[loop right] node{$0,1$} (q8)
                            (q6) edge[above] node{$1$} (q8);
                        \end{tikzpicture}
                    \end{figure}
              \item $\{\epsilon,0\}$
                    \begin{figure}[H]
                        \centering
                        \begin{tikzpicture}
                            \node[state, initial, accepting] (q1) {};
                            \node[state, accepting, above right of=q1] (q2) {};
                            \node[state, below right of=q1] (q3) {};
                            \draw
                            (q1) edge[above] node{$0$} (q2)
                            (q1) edge[right] node{$1$} (q3)
                            (q2) edge[right] node{$0,1$} (q3)
                            (q3) edge[loop right] node{$0,1$} (q3);
                        \end{tikzpicture}
                    \end{figure}
              \item $\{w|w~ \text{contains an even number of }0\text{'s, or contains exactly two }1\text{'s}\}$
                    \begin{figure}[H]
                        \centering
                        \begin{tikzpicture}
                            \node[state, initial, accepting] (q1) {};
                            \node[state, accepting, above right of=q1] (q2) {};
                            \node[state, below right of=q1] (q3) {};
                            \node[state, accepting, right of=q3] (q4) {};
                            \node[state, right of=q2] (q5) {};
                            \node[state, accepting, above right of=q2] (q6) {};
                            \node[state, accepting, right of=q6] (q7) {};
                            \node[state, accepting, above right of=q6] (q8) {};
                            \node[state, right of=q8] (q9) {};
                            \draw
                            (q1) edge[above] node{$1$} (q2)
                            (q1) edge[right] node{$0$} (q3)
                            (q3) edge[below] node{$1$} (q5)
                            (q4) edge[bend right, below] node{$1$} (q2)
                            (q3) edge[bend right, above] node{$0$} (q4)
                            (q4) edge[bend right, below] node{$0$} (q3)
                            (q2) edge[bend right, above] node{$0$} (q5)
                            (q2) edge[right, above] node{$1$} (q6)
                            (q5) edge[right, below] node{$1$} (q7)
                            (q5) edge[bend right, below] node{$0$} (q2)
                            (q6) edge[bend right, above] node{$0$} (q7)
                            (q7) edge[bend right, below] node{$0$} (q6)
                            (q6) edge[right, above] node{$1$} (q8)
                            (q7) edge[right, below] node{$1$} (q9)
                            (q8) edge[bend right, above] node{$0$} (q9)
                            (q9) edge[bend right, below] node{$0$} (q8);
                        \end{tikzpicture}
                    \end{figure}
              \item The empty set
                    \begin{figure}[H]
                        \centering
                        \begin{tikzpicture}
                            \node[state, initial] (q1) {};
                            \draw
                            (q1) edge[loop right] node{$0,1$} (q1);
                        \end{tikzpicture}
                    \end{figure}
              \item All strings except the empty string
                    \begin{figure}[H]
                        \centering
                        \begin{tikzpicture}
                            \node[state, initial] (q1) {};
                            \node[state, accepting, right of=q1] (q2) {};
                            \draw
                            (q1) edge[above] node{$0,1$} (q2)
                            (q2) edge[loop above] node{$0,1$} (q2);
                        \end{tikzpicture}
                    \end{figure}
          \end{enumerate}
\end{enumerate}
