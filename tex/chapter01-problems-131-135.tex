\begin{enumerate}

    \item [1.31]
          For any string $w = w_1w_2\ldots w_n$, the reverse of $w$, written $w^R$, is the string $w$ in reverse order, $w_n \ldots w_2w_1$. For any language A, let $A^R = \{w^R ~|~ w \in A\}$. Show that if $A$ is regular, so is $A^R$.
          \\
          \\
          Let $$D = \{Q, \Sigma, \delta, q_0, F\}$$ be DFA recognizes $A$. We will construct a NFA $$D^R = \{Q^R, \Sigma, \delta^R, q^R_0, \{q_0\}\}$$ that recognizes $A^R$.

          $Q^R = Q \cup \{q^R_0\}$ where $q^R_0$ is new starting state connected with $\epsilon$-edges/transitions to each accepting state of $Q$ from $F$  \\
          $F^R = \{q_0\}$ (starting state from $D$ is accepting state of $D'$) \\
          $\delta^R(q, a) = \{p ~|~ \delta(p, a) = q\}$ \\
          and $\delta^R(q^R_0, \epsilon) = F$.





    \item [1.32]

          Let $\Sigma_3 = \Biggl \{\begin{bmatrix}0 \\ 0 \\ 0\end{bmatrix} , \begin{bmatrix}0 \\0 \\1 \end{bmatrix}, \begin{bmatrix}0 \\1 \\0\end{bmatrix} ,..., \begin{bmatrix}1 \\1 \\1\end{bmatrix} \Biggl \}$ . $\Sigma_3$ contains all size 3 columns of $0$'s and $1$'s. A string of symbols in $\Sigma_3$ gives three rows of $0$'s and $1$'s. Consider each row to be a binary number and let \\
          $B =\{w \in \Sigma^\ast_3 ~|~ \text{the bottom row of }~w~\text{is the sum of the top two rows}\}$. \\
          For example, $ \begin{bmatrix}0 \\0 \\1\end{bmatrix}  \begin{bmatrix}1 \\0 \\0\end{bmatrix} \begin{bmatrix} 1\\ 1\\ 0\end{bmatrix} \in B $, but $\begin{bmatrix}0 \\0 \\1\end{bmatrix} \begin{bmatrix}1\\ 0\\1\end{bmatrix} \notin B$. Show that $B$ is regular. (Hint: Working with $B^R$ is easier. You may assume the result claimed in Problem 1.31.)
          \\
          \\
          Let

          $ 0 = \begin{bmatrix}0 \\ 0 \\ 0\end{bmatrix} $
          $ 1 = \begin{bmatrix}0 \\ 0 \\ 1\end{bmatrix} $
          $ 2 = \begin{bmatrix}0 \\ 1 \\ 0\end{bmatrix} $
          $ \ldots $
          $ 7 = \begin{bmatrix}1 \\ 1 \\ 1\end{bmatrix} $
          be shorthand notation for each element of $\Sigma_3$.

          \begin{figure}[H]
              \centering
              \begin{tikzpicture}[
                  ]
                  \node[state, accepting, initial] (q0) {OK};
                  \node[state, right of=q0] (qacc) {acu};
                  \node[state, below of=qacc] (qbad) {bad};
                  \draw
                  (q0) edge[bend left, above] node{$6$} (qacc)
                  (q0) edge[bend right, below] node{$1,2,4,7$} (qbad)
                  (qacc) edge[bend left, below] node{$1$} (q0)
                  (qacc) edge[right] node{$0,3,5,6$} (qbad)
                  (qacc) edge[loop above, above] node{$2,4,7$} (qacc)
                  (q0) edge[loop above, above] node{$0,3,5$} (q0)
                  (qbad) edge[loop below, below] node{$0,1,2,3,4,5,6,7$} (q0);
              \end{tikzpicture}
              \caption{DFA recognizing $\Sigma^{\ast R}_3 = B^R$}
          \end{figure}

          The DFA above recognizes $B^R$. By Problem 1.31, $B^R$ is regular. Since $B^R$ is regular, $B$ is regular as well.

    \item [1.33]
          Let $\Sigma_2 = \Biggl \{ \begin{bmatrix}0 \\0\end{bmatrix} , \begin{bmatrix}0 \\1\end{bmatrix} , \begin{bmatrix}1\\ 0\end{bmatrix}, \begin{bmatrix}1 \\1\end{bmatrix} \Biggl\}$. Here, $\Sigma_2$ contains all columns of $0$'s and $1$'s of height two. A string of symbols in $\Sigma_2$ gives two rows of $0$'s and $1$'s. Consider each row to be a binary number and let $C =\{w \in \Sigma^\ast_2~|~\text{the bottom row of }w \text{is three times the top row}\}$. For example, $\begin{bmatrix}0 \\0 \end{bmatrix}\begin{bmatrix}0\\ 1 \end{bmatrix}\begin{bmatrix}1 \\1 \end{bmatrix}\begin{bmatrix}0 \\0\end{bmatrix} \in C$, but $\begin{bmatrix}0 \\1\end{bmatrix} \begin{bmatrix}0\\ 1\end{bmatrix} \begin{bmatrix}1\\ 0\end{bmatrix} \notin C$. Show that $C$ is regular. (You may assume the result claimed in Problem 1.31.)

    \item [1.34]

          Let $\Sigma_2$ be the same as in Problem 1.33. Consider each row to be a binary number and let\\
          $D=\{ w \in \Sigma^\ast_2~ |~\text{the top row of }w~ \text{is a larger number than is the bottom row}\}$. \\
          For example, $\begin{bmatrix}0 \\0\end{bmatrix} \begin{bmatrix}1 \\0\end{bmatrix} \begin{bmatrix}1 \\1\end{bmatrix} \begin{bmatrix}0 \\0\end{bmatrix} \in D$, but $\begin{bmatrix}0\\ 0\end{bmatrix} \begin{bmatrix}0\\ 1\end{bmatrix} \begin{bmatrix} 1\\ 1\end{bmatrix} \begin{bmatrix} 0\\ 0\end{bmatrix} \notin D$. \\
          Show that $D$ is regular.
          \\
          \\
          \\
          Let $0 = \begin{bmatrix}0 \\ 0\end{bmatrix}$, $1 = \begin{bmatrix}0 \\ 1\end{bmatrix}$, $2 = \begin{bmatrix}1 \\ 0\end{bmatrix}$, and $3 = \begin{bmatrix}1 \\ 1\end{bmatrix}$ be shorthand notation for each element of $\Sigma_2$ \\
          and assume that $\epsilon \notin D$.

          \begin{figure}[H]
              \centering
              \begin{tikzpicture}[
                  ]
                  \node[state, initial] (q0) {bad};
                  \node[state, accepting, right of=q0] (q1) {OK};
                  \draw
                  (q0) edge[bend left, above] node{$ 2$} (q1)
                  (q0) edge[loop above, above] node{$0, 1, 3$} (q0)
                  (q1) edge[bend left, below] node{$1$} (q0)
                  (q1) edge[loop above, above] node{$0, 2, 3$} (q1);
              \end{tikzpicture}
              \caption{DFA that recognizes $D^R$}
          \end{figure}

          The DFA above recognizes $D^R$. By Problem 1.31, $D$ is regular.

    \item [1.35]
          Let $\Sigma_2$ be the same as in Problem 1.33. Consider the top and bottom rows to be strings of $0$'s  and $1$'s, and let \\
          $E =\{w \in \Sigma^\ast_2 |~ \text{the bottom row of w is the reverse of the top row of w}\}$. \\
          Show that E is not regular.
          \\
          \\
          \\
          Let $e = \begin{bmatrix}0 \\0\end{bmatrix}^p \begin{bmatrix}1 \\1\end{bmatrix} \begin{bmatrix}0 \\0\end{bmatrix}^p$\\
          $\left|e\right| = 2p + 1 \ge p$

          All possible divisions of $e$ into $xyz$ such that $\left|xy\right| \le p$ and $\left|y\right| \ge 1$ are of the form $e = xyz = \left(\begin{bmatrix}0 \\0\end{bmatrix}^k\right) \left(\begin{bmatrix}0 \\0\end{bmatrix}^m\right)\left(\begin{bmatrix}0 \\0\end{bmatrix}^{p-k-m} \begin{bmatrix}1 \\1\end{bmatrix} \begin{bmatrix}0 \\0\end{bmatrix}^{p}\right)$ where $0 \le k \le p$.

          $xy^2z = xyyz = \left(\begin{bmatrix}0 \\0\end{bmatrix}^k\right) \left(\begin{bmatrix}0 \\0\end{bmatrix}^m\right)\left(\begin{bmatrix}0 \\0\end{bmatrix}^m\right)\left(\begin{bmatrix}0 \\0\end{bmatrix}^{p-k-m} \begin{bmatrix}1 \\1\end{bmatrix} \begin{bmatrix}0 \\0\end{bmatrix}^{p}\right) =
              \left(\begin{bmatrix}0 \\0\end{bmatrix}^m\right)\left(\begin{bmatrix}0 \\0\end{bmatrix}^{p} \begin{bmatrix}1 \\1\end{bmatrix} \begin{bmatrix}0 \\0\end{bmatrix}^{p}\right) \notin E$

          Therefore, by pummping lemma, $E$ is not regular.
\end{enumerate}
