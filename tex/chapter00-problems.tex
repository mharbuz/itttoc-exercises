
\begin{enumerate}

    \item[0.10]
Find the error in the following proof that $2 = 1$. 

Consider the equation $a = b$. Multiply both sides by a to obtain 
$a^2 = ab$. 
Subtract 
$b^2$ 
from both sides to get 
$a^2 - b^2 = ab - b^2$
. Now factor each side, 
$(a + b)(a - b) = b(a - b)$
, and divide each side by 
$(a - b)$ 
to get 
$a + b = b$
. Finally, let a and $b$ equal $1$, which shows that $2 = 1$.


We cannot divide each side of equation $(a + b)(a - b) = b(a - b)$ by $(a - b)$ because $a = b$ implies $a - b = 0$ and division by zero is undefined.

    \item[0.11]
Let $S(n) = 1 + 2 + \ldots + n$ be the sum of the first $n$ natural numbers, and let $C(n) = 1^3 + 2^3 + \ldots + n^3$ be the sum of the first n cubes. Prove the following equalities by induction on $n$, to arrive at the curious conclusion that $C(n) = S^2(n)$ for every $n$.

1. $S(n) = \frac{1}{2} n(n+1)$.

\textbf{Basis}
$ S(1) = 1 = \frac{1}{2} 1(1+1)$.

\textbf{Induction step}

$ S(k+1) = S(k) + (k+1) = \frac{1}{2} k(k+1) + (k+1) = \frac{1}{2} (k+1)(k+2)$.

2. $C(n) = \frac{1}{4}(n^4 + 2n^3 + n^2) = \frac{1}{4} n^2(n+1)^2$.

\textbf{Basis}

$ C(1) = 1 = \frac{1}{4} 1^2(1+1)^2$

\textbf{Induction step}

$ C(k+1) = C(k) + (k+1)^3 = \frac{1}{4} k^2(k+1)^2 + (k+1)^3 = \frac{1}{4} (k+1)^2(k+2)^2$.

The induction step is valid for both $S(n)$ and $C(n)$. 

Therefore, $C(n) = S^2(n)$ for every $n$.

$S^2(n)$ is $(\frac{1}{2} n(n+1))^2 = \frac{1}{4} n^2(n+1)^2$ which is equal to $C(n)$.

    \item[0.12]
Find the error in the following proof that all horses are the same color.

\textbf{CLAIM}: In any set of h horses, all horses are the same color.

\textbf{PROOF}: By induction on h.

\textbf{Basis}: 

For $h = 1$. In any set containing just one horse, all horses clearly are the same color.

\textbf{Induction step}

For $k >= 1$, assume that the claim is true for $h = k$ and prove that it is true for $h = k+1$. Take any set $H$ of $k+1$ horses. We show that all the horses in this set are the same color. Remove one horse from this set to obtain the set $H_1$ with just $k$ horses. By the induction hypothesis, all the horses in $H_1$ are the same color. Now replace the removed horse and remove a different one to obtain the set $H_2$. By the same argument, all the horses in $H_2$ are the same color. Therefore, all the horses in $H$ must be the same color, and the proof is complete.

The horse removed in the first step may have a different color than the horse removed in the second step. Therefore, the induction step is invalid.

\item[0.13]
Show that every graph with two or more nodes contains two nodes that have equal degrees.

A graph $G$ is a pair of sets $V$ and $E$ where the elements of the non empty set V are called the vertices and the elements of a possibly empty set E, called the edges, are unordered pairs of vertices.\footnote{https://www.quora.com/How-would-we-prove-that-every-graph-with-at-least-two-vertices-has-two-vertices-of-the-same-degree}

...

Depending on the cardinality of the set V graphs may be finite or infinite. I will only consider finite graphs.

Narrowing the scope further: I shall only consider graphs with no loops and with no multiple edges - in what follows a pair of vertices may be connected with at most one edge.

A vertex $v$ is incident with an edge $e$ if $v$ is a terminal point of $e$. As such, the degree of a vertex $v$ is the number of edges incident with $v$. Intuitively - say, as a vertex I’m a city: the number of distinct roads that emanate from or lead into me is my degree.

Further, since our edge must connect exactly two distinct vertices, $a$ and $b$, it follows that when we count the degree of a connected vertex, $a$ or $b$, we “overcount its edge” (loosely speaking) - that edge contributes $1$ toward the total degree of $a$ and it also contributes $1$ toward the total degree of $b$.
Therefore, the sum of degrees $d_{i}$ (some of which may be zero) of $n$ vertices of a graph must be an even number (in $e$ - the number of edges in a graph):

$$\sum^{n}_{i=1}=1 d_{i}=2e$$

Now onto the theorem - proof by contradiction.

Observe that if the number of vertices in our graph is $n$ then the largest degree of any vertex in such a graph must be strictly less then $n$:

$$max d_{i} \leq n-1$$
for any $i$. Why is that? By contradiction. Recall my roads leaving a city analogy - a road emanating from a given (and fixed) city has only and at most $n - 1$ other cities to go to since:
no loops and
no multiple roads to the same city
are allowed.

Next, assume that, contrary to the conclusion, each and every vertex has a distinct degree. Therefore, our only choices (for these degrees) are:

$D={0,1,2,3,\ldots ,n-1}$

But our graph can not have two vertices with degrees $0$ and $n-1$ simultaneously.

Why is that?

By contradiction - assume that a graph has a vertex $v$ with degree $n-1$ . Then $v$ must find exactly $n-1$ distinct cities on the other side. Therefore, in that case a vertex with degree $0$ can not exist.

Therefore, our set of choices (3) splits into two: either:

$D = \{ 1 , 2 , 3 , 4 , \ldots ,n - 1 \}$

or:

$D = \{ 0, 1, 2, 3, \ldots , n-2\}$


But in either case the cardinality of $D$ is $n-1$:

$|D|=n-1$

Therefore, we have to distribute $n$ vertices over $n-1$ degrees and by pigeonhole principle, since we have more vertices (pigeons) than degrees (pigeonholes):

at least two vertices must share the same degree

which amounts to a contradiction - our initial assumption that all the degrees are distinct was false. Therefore, if our graph has at least two vertices then at least two of them have the same degree.

\item[0.14]
Ramsey's theorem. 

Let $G$ be a graph. A clique in $G$ is a subgraph in which every two nodes are connected by an edge. An anti-clique, also called an independent set, is a subgraph in which every two nodes are not connected by an edge. Show that every graph with $n$ nodes contains either a clique or an anti-clique with at least $\frac{1}{2} log_2(n)$ nodes.


\item[0.15]
Use Theorem 0.25 to derive a formula for calculating the size of the monthly payment for a mortgage in terms of the principal P, the interest rate I, and the number of payments t. Assume that after t payments have been made, the loan amount is reduced to 0. Use the formula to calculate the dollar amount of each monthly payment for a 30-year mortgage with 360 monthly payments on an initial loan amount of \$100,000 with a 5\% annual interest rate.


\textbf{Theorem 0.25}

$\forall t \geq 0$


$P_t=PM^t-Y\frac{M^t-1}{M-1}$
\end{enumerate}