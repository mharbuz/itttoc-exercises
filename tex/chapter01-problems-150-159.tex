\begin{enumerate}

      \item [1.50]
            
            Read the informal definition of the finite state transducer given in Exercise 1.24. Prove that no FST can output $w^R$ for every input $w$ if the input and output alphabets are $\{0,1\}$.
            
      \item [1.51]
            
            Let $x$ and $y$ be strings and let $L$ be any language. We say that $x$ and $y$ are \textbf{distinguishable by $L$} if some string $z$ exists whereby exactly one of the strings $xz$ and $yz$ is a member of $L$; otherwise, for every string $z$, we have $xz \in L$ whenever $yz \in L$ and we say that $x$ and $y$ are \textbf{indistinguishable by $L$}. 
            
            If $x$ and $y$ are indistinguishable by $L$, we write $x \equiv_{L} y$ . Show that $\equiv_{L}$ is an equivalence relation.
            
      \item [1.52]
            
            \textbf{Myhill–Nerode theorem}. 
            
            Refer to Problem 1.51. Let $L$ be a language and let $X$ be a set of strings. Say that $X$ is \textbf{pairwise distinguishable by L} if every two distinct strings in $X$ are distinguishable by L. Define the index of $L$ to be the maximum number of elements in any set that is pairwise distinguishable by $L$. The index of $L$ may befinite or infinite. 
            
            \begin{enumerate}
                  \item Show that if $L$ is recognized by a $DFA$ with $k$ states, $L$ has index at most $k$.
                  \item Show that if the index of $L$ is a finite number $k$, it is recognized by a $DFA$ with $k$ states.
                  \item Conclude that $L$ is regular iff it has finite index. Moreover, its index is the size of the smallest $DFA$ recognizing it.
            \end{enumerate}
            
      \item [1.53]
            
            Let $\Sigma = \{0,1,+,=\}$ and 
            
            $$ADD = \{x=y+z~|~x,y,z~\text{are binary integers, and}~ x~ \text{is the sum of}~ y ~\text{and}~ z\}.$$
            
            Show that $ADD$ is not regular.
            
      \item [1.54]
            
            Consider the language $F = \{a^i b^j c^k ~|~ i,j,k \geq 0~ \text{and if}~ i=1~ \text{then}~~ j = k\}$. 
            \begin{enumerate}
                  \item Show that $F$ is not regular.
                  \item Show that $F$ acts like a regular language in the pumping lemma. In other words, give a pumping length $p$ and demonstrate that $F$ satisfies the three conditions of the pumping lemma for this value of $p$.
                  \item Explain why parts (a) and (b) do not contradict the pumping lemma.
                        
            \end{enumerate}
            
      \item [1.55]
            
            The pumping lemma says that every regular language has a pumping length $p$, such that every string in the language can be pumped if it has length $p$ or more. If $p$ is a pumping length for language $A$, so is any length $p' \ge p$.The minimum pumping length for $A$ is the smallest $p$ that is a pumping length for $A$. For example, if $A=01^\ast$,the minimum pumping length is 2. The reason is that the string $s = 0$ is in $A$ and has length $1$ yet $s$ cannot be pumped; but any string in $A$ of length $2$ or more contains a $1$ and hence can be pumped by dividing it so that $x = 0, y = 1$, and $z$ is the rest. For each of the following languages, give the minimum pumping length and justify your answer.
            
            \begin{enumerate}
                  \item $0001^\ast$
                  \item $0^\ast 1^\ast$
                  \item $001 \cup 0^\ast 1^\ast$
                  \item $0^\ast 1^+ 0^+ 1^\ast \cup 10^\ast1$
                  \item $(01)^\ast$
                  \item $\epsilon$
                  \item $1^\ast01^\ast01^\ast$
                  \item $10(11^\ast0)^\ast0$
                  \item $1011$
                  \item $\Sigma^\ast$
            \end{enumerate}
            
      \item [1.56]
            
            If $A$ is a set of natural numbers and $k$ is a natural number greater than $1$, let $B_k(A)=\{w~|~w~ \text{is the representation in base}~k~\text{of some number in} A\}$. Here, we do not allow leading $0$s in the representation of a number. For example, $B_2({3,5})=\{11,101\}$ and $B_3({3,5})=\{10,12\}$. Give an example of a set $A$ for which $B_2(A)$ is regular but $B_3(A)$ is not regular. 
            
            Prove that your example works.
            
      \item [1.57]
            
            If $A$ is any language, let $A_{\frac{1}{2}-}$ be the set of all first halves of strings in A so that $A_{\frac{1}{2}-} = \{x~|~ \text{for some}~ y, |x| = |y| ~\text{and}~ xy \in A\}$. 
            
            Show that if $A$ is regular, then so is $A_{\frac{1}{2}-}$.
            
      \item [1.58]

      If $A$ is any language, let $A_{\frac{1}{3}-\frac{1}{3}}$ be the set of all strings in A with their middle thirds removed so that $A_{\frac{1}{3}-\frac{1}{3}} = \{xz~|~ \text{for some}~ y, |x| = |y| = |z|~ \text{and}~ xyz \in A\}$. Show that if $A$ is regular, then $A_{\frac{1}{3}-\frac{1}{3}}$ is not necessarily regular.

      \item [1.59]

      Let $M = (Q,\Sigma, \delta, q_0, F)$ be a DFA and let $h$ be a state of $M$ called its “home”. A synchronizing sequence for $M$ and $h$ is a string $s \in \Sigma^\ast$ where $\delta(q,s)=h$ for every $q \in Q$. 
      
      (Here we have extended $\delta$ to strings, so that $\delta(q,s)$ equals the state where $M$ ends up when $M$ starts at state $q$ and reads input $s$.) 
      
      Say that $M$ is synchronizable if it has a synchronizing sequence for some state $h$. Prove that if $M$ is a $k$-state synchronizable DFA, then it has a synchronizing sequence of length at most $k^3$. 
      
      Can you improve upon this bound?
            
\end{enumerate}
